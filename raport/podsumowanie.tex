\section{Podsumowanie}
\subsection{Stopień zgodności z wymaganiami}
\subsubsection{Wymagania funkcjonalne}
\begin{enumerate}
  \item{\hyperref[func_1]{\textcolor{green}{Zrealizowane} - odpowiedni przycisk w terminalu po wybraniu trasy}}
  \item{\hyperref[func_2]{\textcolor{green}{Zrealizowane} - wybór za pomocą enkodera}}
  \item{\hyperref[func_3]{\textcolor{green}{Zrealizowane} - realizowane poprzez przyciski}}
  \item{\hyperref[func_4]{\textcolor{green}{Zrealizowane}}}
  \item{\hyperref[func_5]{\textcolor{green}{Zrealizowane} - należy przyłożyć kartę przy wsiadaniu i wysiadaniu}}
  \item{\hyperref[func_6]{\textcolor{green}{Zrealizowane} - koszt wynosi 1}}
  \item{\hyperref[func_7]{\textcolor{green}{Zrealizowane} - panel administratora}}
  \item{\hyperref[func_8]{\textcolor{green}{Zrealizowane} - panel administratora}}
  \item{\hyperref[func_9]{\textcolor{green}{Zrealizowane} - panel administratora}}
\end{enumerate}
\subsubsection{Wymagania niefunkcjonalne}
\begin{enumerate}
  \item{\hyperref[nfunc_1]{\textcolor{green}{Zrealizowane}}}
  \item{\hyperref[nfunc_2]{\textcolor{green}{Zrealizowane}}}
\end{enumerate}
\subsection{Napotkane problemy w implementacji}
\subsubsection{Uruchomienie systemu na komputerze laboratoryjnym}
Na komputerach laboratoryjnych nie ma zainstalowanego \verb|Node.js|, a użytkownik studenta nie ma uprawnień administratora, więc też nie było możliwości instalacji. Problem dotyczył panelu administracyjnego i został rozwiązany poprzez zmianę środowiska na Raspberry Pi, tam już instalacja była możliwa.
