\section{Wymagania projektowe}
\subsection{Podstawowe wymagania funkcjonalne}
\subsubsection{Kierowca}
\begin{enumerate}
  \item{Jako kierowca chcę zmieniać przystanki za pomocą terminalu w trakcie wykonywania kursu}\label{func_1}
  \item{Jako kierowca chcę wybierać trasę kursu w terminalu przy rozpoczynaniu jazdy}\label{func_2}
  \item{Jako kierowca chcę za pomocą terminalu rozpoczynać jazdę i ją kończyć po przejechaniu trasy}\label{func_3}
  \item{Jako kierowca chcę, aby po dotarciu na ostatni przystanek jazda kończyła się automatycznie}\label{func_4}
\end{enumerate}
\subsubsection{Pracownik}
\begin{enumerate}[resume]
  \item{Jako pracownik chcę wsiadać na przystanku i płacić za przejazd za pomocą swojej karty pracowniczej}\label{func_5}
  \item{Jako pracownik chcę, aby koszt przejazdu był zależny od przejechanych przystanków i był pobierany dopiero po zakończeniu przejazdu (przyłożeniu karty pracowniczej drugi raz)}\label{func_6}
\end{enumerate}
\subsubsection{Administrator}
\begin{enumerate}[resume]
  \item{Jako administrator chcę zarządzać\footnote{Poprzez zarządzanie rozumiemy operacje: dodawania, odczytu, aktualizowania i usuwania} wszystkimi kursami}\label{func_7}
  \item{Jako administrator chcę zarządzać wszystkimi przystankami}\label{func_8}
  \item{Jako administrator chcę zarządzać wszystkimi pracownikami, w szczególności ilością pieniędzy na ich koncie}\label{func_9}
\end{enumerate}
\subsection{Podstawowe wymagania niefunkcjonalne}
\begin{enumerate}
  \item{System powinien obsługiwać więcej niż jednego pracownika naraz}\label{nfunc_1}
  \item{Operacje wykonywane na terminalu (płatność przez pracownika, zmiana trasy i przystanków przez kierowcę) powinny być jak najszybciej przekazywane do bazy danych}\label{nfunc_2}
\end{enumerate}
