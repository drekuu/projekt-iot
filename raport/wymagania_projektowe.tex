\section{Wymagania projektowe}
\subsection{Podstawowe wymagania funkcjonalne}
\subsubsection{Kierowca}
\begin{enumerate}
  \item{Jako kierowca chcę zmieniać przystanki za pomocą terminalu w trakcie wykonywania kursu}
  \item{Jako kierowca chcę wybierać trasę kursu w terminalu przy rozpoczynaniu jazdy}
  \item{Jako kierowca chcę za pomocą terminalu rozpoczynać jazdę i ją kończyć na dowolnym przystanku}
  \item{Jako kierowca chcę, aby po dotarciu na ostatni przystanek jazda kończyła się automatycznie}
\end{enumerate}
\subsubsection{Pracownik}
\begin{enumerate}[resume]
  \item{Jako pracownik chcę wsiadać na przystanku i płacić za przejazd za pomocą swojej karty pracowniczej}
  \item{Jako pracownik chcę, aby koszt przejazdu był zależny od przejechanych przystanków i był pobierany dopiero po zakończeniu przejazdu (przyłożeniu karty pracowniczej drugi raz)}
\end{enumerate}
\subsubsection{Administrator}
\begin{enumerate}[resume]
  \item{Jako administrator chcę zarządzać\footnote{Poprzez zarządzanie rozumiemy operacje: dodawania, odczytu, aktualizowania i usuwania} wszystkimi kursami}
  \item{Jako administrator chcę zarządzać wszystkimi przystankami}
  \item{Jako administrator chcę zarządzać wszystkimi pracownikami, w szczególności ilością pieniędzy na ich koncie}
\end{enumerate}
\subsection{Podstawowe wymagania niefunkcjonalne}
\begin{enumerate}
  \item{System powinien obsługiwać więcej niż jednego pracownika naraz}
  \item{System powinien obsługiwać wiele pojazdów (w tym terminalów) naraz}
  \item{Operacje wykonywane na terminalu (płatność przez pracownika, zmiana trasy i przystanków przez kierowcę) powinny być jak najszybciej przekazywane do bazy danych}
\end{enumerate}
