\section{Wstęp}
\subsection{Problem do rozwiązania}
W firmie przewozowej komunikacji miejskiej postanowiono ułatwić korzystanie z systemu i połączyć pewne funkcjonalności z Internetem. Do tej pory, pojazdy informacje o kolejnych przystankach i trasach komunikowały tylko lokalnie w pojeździe, a pracownicy którzy płacili za przejazd mogli to robić tylko gotówką.
\subsection{Krótki opis implementacji}
Do rozwiązania problemu stworzono system który:
\begin{itemize}
  \item{umożliwia pracownikowi płatność za pomocą karty pracowniczej, którą można zasilać środkami,}
  \item{umożliwia kierowcy pojazdu zmianę przystanku na którym się znajduje, jak i trasy po której ma zamiar odbyć kurs,}
  \item{umożliwia administratorowi systemu zarządzanie danymi w bazie.}
\end{itemize}
Część systemu która znajduje się w pojeździe jest nazywana terminalem. \\
Terminal jest obsługiwany przez kierowcę i pracownika. \\
Terminal posiada wyświetlacz, czytnik kart RFID, pasek LED, dwa przyciski i enkoder. \\
Do bazy danych przekazywane są dane z terminala (zmiana przystanku bądź trasy, płatność).
