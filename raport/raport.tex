\documentclass[10pt, a4paper, titlepage]{article}
\usepackage[cm]{fullpage}
\usepackage[polish]{babel}
\usepackage[utf8]{inputenc}
\usepackage[T1]{fontenc}
\usepackage{hyperref}
\usepackage{csquotes}
\usepackage{float}
\usepackage{graphicx}
\usepackage{cprotect}
\usepackage{tikz}
\usepackage{varwidth}
\usepackage{amsmath}
\usepackage{titling}
\usetikzlibrary{positioning}
\graphicspath{ {./images/} }
\hypersetup{
  colorlinks=true,
  linktoc=all,
  linkcolor=black,
}
\setcounter{tocdepth}{3}
\setcounter{secnumdepth}{3}

\newcommand{\subtitle}[1]{%
  \posttitle{%
    \par\end{center}
    \begin{center}\large#1\end{center}
    \vskip0.5em}%
}

\title{Projekt programistyczny}
\subtitle{
  Podstawy internetu rzeczy \\
  Prowadzący laboratorium: dr inż. Krzysztof Chudzik
}
\author{
  Kacper Gaudyn, nr indeksu 266873 \\
  Kuba Krąpiec, nr indeksu 266503 \\
  Michał Pesta, nr indeksu 266899 \\
  Michał Trojanowski, nr indeksu 266864
}
\date{Data ukończenia pracy: ...}

\begin{document}
  \begin{titlingpage}
    \maketitle
  \end{titlingpage}
  \tableofcontents

  \newpage
  \section{Wymagania projektowe}
\subsection{Podstawowe wymagania funkcjonalne}
\subsection{Podstawowe wymagania niefunkcjonalne}

  \section{Opis architektury systemu}
\subsection{Elementy architektury z opisem}
\begin{itemize}
  \item{Panel administracyjny - służy administratorom do zarządzania danymi}
  \item{Baza danych - zawiera wszystkie dane}
  \item{REST API - zarządza danymi bazy danych i udostępnia odpowiednie operacje innym podmiotom}
  \item{Broker MQTT - komunikuje się z terminalami w pojazdach i przekazuje informacje o przejechanych przystankach i płatnościach}
  \item{Terminal - znajduje się w każdym pojeździe, umożliwia kierowcy ustalanie trasy i zmianę przystanków, oraz umożliwia pracownikom płacenie za swoje przejazdy}
  \item{
    Użytkownicy
    \begin{itemize}
      \item{Kierowca - kieruje pojazdem i na terminalu może zmieniać trasy, przystanki}
      \item{Pracownik - może wsiadać do pojazdów i płacić w terminalu za przejazd kartą}
      \item{Administrator - zarządza danymi w systemie}
    \end{itemize}
  }
\end{itemize}
\subsection{Graficzna reprezentacja architektury}
\begin{figure}[H]
  \centering
  \begin{tikzpicture}[>=Latex,
    user/.style={draw=gray,dashed}
  ]
    \node at (0,0) [rectangle,draw] (database) {Baza danych};
    \node [below=of database,rectangle,draw] (rest) {REST API};
    \node [below left=of rest,rectangle,draw] (frontend_admin) {Panel administracyjny};
    \node [below right=of rest,rectangle,draw] (broker) {Broker MQTT};
    \node [below=of broker,rectangle,draw] (terminal) {Terminal};
    \node [below left=of broker,rectangle] (terminaldotsl) {\dots};
    \node [below right=of broker,rectangle] (terminaldotsr) {\dots};

    \node[below=0.5 of frontend_admin,rectangle,draw,user] (admin) {Administrator};
    \node[below left=0.5 of terminal,rectangle,draw,user] (driver) {Kierowca};
    \node[below right=0.5 of terminal,rectangle,draw,user] (worker) {Pracownik};

    \draw [<->] (database) -- (rest);
    \draw [<->] (rest) -- (frontend_admin);
    \draw [<->] (rest) -- (broker);
    \draw [<->] (broker) -- (terminal);
    \draw [<->] (broker) -- (terminaldotsl);
    \draw [<->] (broker) -- (terminaldotsr);

    \draw [<->] (admin) -- (frontend_admin);
    \draw [<->] (driver) -- (terminal);
    \draw [<->] (worker) -- (terminal);
  \end{tikzpicture}
  \caption{Diagram elementów architektury z kierunkami przekazywania danych}
\end{figure}
\subsection{Baza danych}
\subsubsection{Schemat bazy danych}
\begin{figure}[H]
  \centering
  \begin{tikzpicture}[>=Latex,
    db/.style={
      draw, matrix of nodes,
      nodes={
        node family/text width/.expanded=%
          \tikzmatrixname-\the\pgfmatrixcurrentcolumn,
        node family/text width align=left,
        inner xsep=+.5\tabcolsep, inner ysep=+0pt, align=left},
      inner sep=.5\pgflinewidth,
      font=\strut\ttfamily,
    }
  ]
    \matrix[db,label=Courses,row 1/.style={nodes={fill=red!20}}] (courses) {
      CourseID & int & \node (courses_courseid) {PK}; \\
      CourseName & varchar(255) \\
    };
    \matrix[db,label=Stops,below=of courses,row 1/.style={nodes={fill=red!20}}] (stops) {
      StopID & int & \node (stops_stopid) {PK}; \\
      StopName & varchar(255) \\
    };
    \matrix[db,label=Assignments,below=of stops,row 1/.style={nodes={fill=blue!20}},row 2/.style={nodes={fill=blue!20}}] (assignments) {
      CourseID & int & \node (assignments_courseid) {FK}; \\
      StopID & int & \node (assignments_stopid) {FK}; \\
      StopNumber & int \\
    };
    \matrix[db,label=Workers,below=of assignments,row 1/.style={nodes={fill=red!20}}] (workers) {
      WorkerID & int & \node (workers_workerid) {PK}; \\
      WorkerFirstName & varchar(255) \\
      WorkerLastName & varchar(255) \\
      WorkerBalance & int \\
      WorkerCardID & varchar(255) \\
    };
    \matrix[db,label=CurrentRides,below=of workers,row 1/.style={nodes={fill=red!20}},row 2/.style={nodes={fill=blue!20}}] {
      RideID & int & PK \\
      WorkerID & int & \node (currentrides_workerid) {FK}; \\
      StopsTraveled & int \\
    };
    \draw [<-] (courses_courseid.east) -- ++(0.5,0) |- (assignments_courseid.east);
    \draw [<-] (stops_stopid.east) -- ++(0.5,0) |- (assignments_stopid.east);
    \draw [<-] (workers_workerid.east) -- ++(0.5,0) |- (currentrides_workerid.east);
  \end{tikzpicture}
  \caption{Tabele bazy danych z zaznaczonymi relacjami między nimi}
\end{figure}
\subsubsection{Scenariusze i ich wpływ na dane}
\begin{enumerate}
  \item{
    Pracownik wsiada na przystanku, przykłada kartę i po przejechaniu 3 przystanków przykłada ją znowu aby zapłacić i wysiada. \\
    Zakładamy, że każdy przystanek kosztuje 10, czyli pracownik zapłaci 30.
    \begin{figure}[H]
      \centering
      \begin{tikzpicture}[>=Latex,
        db/.style={
          draw, matrix of nodes,
          nodes={
            node family/text width/.expanded=%
              \tikzmatrixname-\the\pgfmatrixcurrentcolumn,
            node family/text width align=left,
            inner xsep=+.5\tabcolsep, inner ysep=+0pt, align=left},
          inner sep=.5\pgflinewidth,
          font=\strut\ttfamily,
          row 1/.style={nodes={fill=gray!20}}
        }
      ]
        \matrix[db,label=Workers] (workers) {
          WorkerID & WorkerFirstName & WorkerLastName & WorkerBalance & WorkerCardID \\
          1 & Jan & Kowalski & 100 & ABC \\
        };
      \end{tikzpicture}
      \caption{Dane w bazie przed wykonaniem scenariusza}
    \end{figure}
    \begin{figure}[H]
      \centering
      \begin{tikzpicture}[>=Latex,
        db/.style={
          draw, matrix of nodes,
          nodes={
            node family/text width/.expanded=%
              \tikzmatrixname-\the\pgfmatrixcurrentcolumn,
            node family/text width align=left,
            inner xsep=+.5\tabcolsep, inner ysep=+0pt, align=left},
          inner sep=.5\pgflinewidth,
          font=\strut\ttfamily,
          row 1/.style={nodes={fill=gray!20}}
        }
      ]
        \matrix[db,label=Workers] (workers) {
          WorkerID & WorkerFirstName & WorkerLastName & WorkerBalance & WorkerCardID \\
          1 & Jan & Kowalski & 70 & ABC \\
        };
        \matrix[db,label=CurrentRides,below=of workers] {
          RideID & WorkerID & StopsTraveled \\
          \vdots \\
          5 & 1 & 3 \\
          \vdots \\
        };
      \end{tikzpicture}
      \caption{Dane w bazie po wykonaniu scenariusza}
    \end{figure}
    W tabeli \verb|Workers| zmieniła się kolumna \verb|WorkerBalance|, a w tabeli \verb|CurrentRides| został dodany nowy rekord.
  }
\end{enumerate}

  \section{Opis implementacji i zastosowanych rozwiązań}
\subsection{Najważniejsze funkcje systemu}
\subsection{Implementacja MQTT}
\subsection{Implementacja szyfrowania i uwierzytelniania}

  \section{Opis działania i prezentacja interfejsu}
\subsection{Sposób instalacji i uruchomienia aplikacji}
\subsubsection{Zawartość plików}
\begin{itemize}
  \item{
    \verb|backend/|
    \begin{itemize}
      \item{baza danych - pliki: \verb|db_management.py|, \verb|init.sql|, \verb|insert_example.sql|}
      \item{REST API - plik: \verb|server.py|}
      \item{broker MQTT - pliki: \verb|mqtt_server.py|, \verb|mqtt_client_TEST.py|}
    \end{itemize}
  }
  \item{\verb|frontend/app| - panel administracyjny}
  \item{\verb|client/| - terminal}
\end{itemize}
\subsubsection{Instalacja}
\begin{enumerate}
  \item{
    Baza danych
    \begin{enumerate}
      \item{przejdź do folderu \verb|backend|}
      \item{uruchom plik \verb|db_management.py|}
      \item{w folderze powinien utworzyć się plik \verb|alpha.db|, zawierający przykładowe dane}
    \end{enumerate}
  }
  \item{
    REST API i broker MQTT
    \begin{enumerate}
      \item{zainstaluj wymagane pakiety za pomocą komendy \verb|pip install -r requirements.txt|}
      \item{
        upewnij się, że masz zainstalowane i uruchomione \href{https://mosquitto.org/download/}{Mosquitto} \\
        UWAGA: w pliku konfiguracyjnym Mosquitto powinny się znaleźć następujące linijki, aby urządzenia mogły się połączyć:
        \begin{lstlisting}[caption={Dodatek do pliku \texttt{mosquitto.conf}}]
        allow_anonymous true
        listener 1883 0.0.0.0
        \end{lstlisting}
      }
      \item{uruchom plik \verb|server.py|}
      \item{działanie REST API można testować pod adresem: \url{http://localhost:55555/docs}}
    \end{enumerate}
  }
  \item{
    Panel administracyjny
    \begin{enumerate}
      \item{upewnij się, że masz zainstalowany \href{https://nodejs.org/en}{Node.js}}
      \item{przejdź do folderu \verb|frontend/app|}
      \item{zainstaluj pakiety za pomocą komendy \verb|npm install|}
      \item{wprowadź komendę \verb|npm run dev|}
      \item{działanie można testować pod adresem: \url{http://localhost:5173/}}
    \end{enumerate}
  }
  \item{
    Terminal (wersja Raspberry Pi)
    \begin{enumerate}
      \item{przejdź do folderu \verb|client|}
      \item{otwórz plik \verb|client.py| i zmodyfikuj zmienną \verb|server_ip| tak, aby odpowiadała adresowi IP brokera MQTT}
      \item{uruchom plik \verb|client.py|}
    \end{enumerate}
  }
  \item{
    Terminal (wersja testowa)
    \begin{enumerate}
      \item{przejdź do folderu \verb|backend|}
      \item{otwórz plik \verb|mqtt_client_TEST.py| i zmodyfikuj zmienną \verb|BROKER_IP| tak, aby odpowiadała adresowi IP brokera MQTT}
      \item{uruchom plik \verb|mqtt_client_TEST.py|}
    \end{enumerate}
  }
\end{enumerate}
\subsection{Przedstawienie działania aplikacji}
\subsubsection{Panel administracyjny}
\begin{figure}[H]
  \centering
  \includegraphics[width=1\textwidth]{admin-panel/employees}
  \caption{Widok pracowników w panelu administracyjnym}
\end{figure}
Przykładowo dodamy nowego pracownika i dodamy mu pieniądze do stanu konta, wypełniając odpowiednie pola i klikając znak \verb|+|.
\begin{figure}[H]
  \centering
  \includegraphics[width=1\textwidth]{admin-panel/employees-add}
  \caption{Dodany nowy pracownik}
\end{figure}
\begin{figure}[H]
  \centering
  \includegraphics[width=0.75\textwidth]{admin-panel/routes-course}
  \caption{Widok przejazdów w panelu administracyjnym, kursy}
\end{figure}
\begin{figure}[H]
  \centering
  \includegraphics[width=0.5\textwidth]{admin-panel/routes-stops}
  \caption{Widok przejazdów w panelu administracyjnym, przystanki}
\end{figure}
\begin{figure}[H]
  \centering
  \includegraphics[width=0.5\textwidth]{admin-panel/routes-add}
  \caption{Widok przejazdów w panelu administracyjnym, dodawanie nowych danych}
\end{figure}
Przykładowo dodamy nowy przystanek o nazwie \verb|Test Stop|, oraz kurs o nazwie \verb|TestCourse|, który będzie go zawierał.
\begin{figure}[H]
  \centering
  \includegraphics[width=0.75\textwidth]{admin-panel/routes-new}
  \caption{Dodany nowy kurs zawierający nowy przystanek}
\end{figure}
\begin{figure}[H]
  \centering
  \includegraphics[width=1\textwidth]{admin-panel/map}
  \caption{Mapa pojazdów odbywających kursy}
\end{figure}
W terminalu zmieniamy przystanek na którym pojazd się znajduje.
\begin{figure}[H]
  \centering
  \includegraphics[width=1\textwidth]{admin-panel/map-nextstop}
  \caption{Zmiana przystanku widoczna na mapie}
\end{figure}
\subsubsection{Terminal (wersja Raspberry Pi)}
\begin{figure}[H]
  \centering
  \includegraphics[width=0.4\textwidth]{terminal-pi/course}
  \caption{Widok wyboru trasy}
\end{figure}
\begin{figure}[H]
  \centering
  \includegraphics[width=0.4\textwidth]{terminal-pi/stop}
  \caption{Widok aktualnego przystanku na trasie}
\end{figure}
\subsubsection{Terminal (wersja testowa)}
\begin{figure}[H]
  \centering
  \includegraphics[width=0.4\textwidth]{terminal-test/menu}
  \caption{Widok menu}
\end{figure}
\begin{figure}[H]
  \centering
  \includegraphics[width=0.6\textwidth]{terminal-test/example}
  \caption{Przykład wyboru kursu, zmiany przystanków i płacenia za przejazd przez pracownika}
\end{figure}

  \section{Wkład pracy Autorów}
\subsection{Pierwszy tydzień (8.01 - 14.01)}
\begin{itemize}
  \item{
    Kacper Gaudyn
    \begin{itemize}
      \item{struktura bazy danych, tabele: \verb|Courses|, \verb|Stops|, \verb|Assignments|, \verb|Workers|, \verb|CurrentRides|}
      \item{tworzenie pliku bazy danych, podstawowe operacje na danych}
    \end{itemize}
  }
  \item{
    Jakub Krąpiec
    \begin{itemize}
      \item{terminal - obsługa przycisków, struktura przystanków i tras}
      \item{dodanie możliwości wyboru trasy i przystanków za pomocą enkodera, obsługa przycisków rozbudowana o nowe funkcjonalności, obsługa RFID}
    \end{itemize}
  }
  \item{
    Michał Pesta
    \begin{itemize}
      \item{utworzenie szkieletu panelu administracyjnego, dodanie paska nawigacji}
      \item{dodanie strony z informacjami o pracownikach}
    \end{itemize}
  }
  \item{
    Michał Trojanowski
    \begin{itemize}
      \item{utworzenie szkieletu raportu, dodanie do raportu wymagań projektowych i opisu architektury systemu}
      \item{terminal - wyświetlanie informacji na wyświetlaczu}
    \end{itemize}
  }
\end{itemize}
\subsection{Drugi tydzień (15.01 - 21.01)}
\begin{itemize}
  \item{
    Kacper Gaudyn
    \begin{itemize}
      \item{dodanie szkieletu REST API}
      \item{dodanie do bazy danych tabeli \verb|Buses|, pobieranie pracowników i pojazdów w REST API, drobne poprawki}
      \item{dodanie brokera MQTT, dokumentacji do brokera MQTT oraz terminala w wersji testowej}
    \end{itemize}
  }
  \item{
    Jakub Krąpiec
    \begin{itemize}
      \item{terminal - poprawki w obsłudze przycisków i enkodera}
      \item{poprawki w strukturze bazy danych, dodano pobieranie kursów w REST API}
      \item{terminal - zmiany wizualne w wyświetlaniu informacji}
    \end{itemize}
  }
  \item{
    Michał Pesta
    \begin{itemize}
      \item{dodanie przykładowych danych do bazy danych, dodana funkcjonalność zmiany przystanków w REST API}
      \item{pobieranie danych z REST API w panelu administratora, dodanie podglądu pojazdów w ruchu}
      \item{dodanie dokumentacji do REST API, drobne poprawki w bazie danych i REST API, dodanie funkcjonalności związanych z pracownikami w REST API}
    \end{itemize}
  }
  \item{
    Michał Trojanowski
    \begin{itemize}
      \item{aktualizacja architektury systemu w raporcie, dodanie do raportu częściowego opisu implementacji, dodanie do raportu wstępu}
      \item{terminal - dodana komunikacja z brokerem MQTT}
      \item{terminal - wyświetlanie informacji od brokera}
    \end{itemize}
  }
\end{itemize}
\subsection{Trzeci tydzień (22.01 - 26.01)}
\begin{itemize}
  \item{
    Kacper Gaudyn
    \begin{itemize}
      \item{poprawki w testowych danych bazy danych, poprawki w obsłudze identyfikatora karty pracownika w brokerze MQTT}
      \item{drobne poprawki w REST API i dokumentacji}
      \item{dokończenie opisu architektury w raporcie}
    \end{itemize}
  }
  \item{
    Jakub Krąpiec
    \begin{itemize}
      \item{drobne poprawki w raporcie, zmiany w komunikacji z brokerem MQTT w terminalu, pobieranie danych o trasach i przystankach z REST API w terminalu}
      \item{drobne poprawki w terminalu}
      \item{dodanie opisu implementacji w raporcie}
    \end{itemize}
  }
  \item{
    Michał Pesta
    \begin{itemize}
      \item{obsługa przystanków w panelu administracyjnym; komunikacja z REST API, poprawki w ustawieniach CORS}
      \item{drobne poprawki w REST API oraz panelu administracyjnym}
      \item{dodanie opisu działania w raporcie}
    \end{itemize}
  }
  \item{
    Michał Trojanowski
    \begin{itemize}
      \item{obsługa kursów w panelu administracyjnym}
      \item{drobne poprawki w REST API}
      \item{dodanie do raportu: wkładu pracy i podsumowania, zrobienie prezentacji}
    \end{itemize}
  }
\end{itemize}

  \section{Podsumowanie}
\subsection{Stopień zgodności z wymaganiami}
\subsubsection{Wymagania funkcjonalne}
\begin{enumerate}
  \item{\hyperref[func_1]{\textcolor{green}{Zrealizowane} - odpowiedni przycisk w terminalu po wybraniu trasy}}
  \item{\hyperref[func_2]{\textcolor{green}{Zrealizowane} - wybór za pomocą enkodera}}
  \item{\hyperref[func_3]{\textcolor{green}{Zrealizowane} - realizowane poprzez przyciski}}
  \item{\hyperref[func_4]{\textcolor{green}{Zrealizowane}}}
  \item{\hyperref[func_5]{\textcolor{green}{Zrealizowane} - należy przyłożyć kartę przy wsiadaniu i wysiadaniu}}
  \item{\hyperref[func_6]{\textcolor{green}{Zrealizowane} - koszt wynosi 1}}
  \item{\hyperref[func_7]{\textcolor{green}{Zrealizowane} - panel administratora}}
  \item{\hyperref[func_8]{\textcolor{green}{Zrealizowane} - panel administratora}}
  \item{\hyperref[func_9]{\textcolor{green}{Zrealizowane} - panel administratora}}
\end{enumerate}
\subsubsection{Wymagania niefunkcjonalne}
\begin{enumerate}
  \item{\hyperref[nfunc_1]{\textcolor{green}{Zrealizowane}}}
  \item{\hyperref[nfunc_2]{\textcolor{green}{Zrealizowane}}}
\end{enumerate}
\subsection{Napotkane problemy w implementacji}
\subsubsection{Uruchomienie systemu na komputerze laboratoryjnym}
Na komputerach laboratoryjnych nie ma zainstalowanego \verb|Node.js|, a użytkownik studenta nie ma uprawnień administratora, więc też nie było możliwości instalacji. Problem dotyczył panelu administracyjnego i został rozwiązany poprzez zmianę środowiska na Raspberry Pi, tam już instalacja była możliwa.

  \section{Literatura}

\end{document}
