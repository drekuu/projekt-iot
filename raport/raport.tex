\documentclass[10pt, a4paper, titlepage]{article}
\usepackage[cm]{fullpage}
\usepackage[polish]{babel}
\usepackage[utf8]{inputenc}
\usepackage[T1]{fontenc}
\usepackage{hyperref}
\usepackage{csquotes}
\usepackage{float}
\usepackage{graphicx}
\usepackage{cprotect}
\usepackage{tikz}
\usepackage{varwidth}
\usepackage{amsmath}
\usepackage{titling}
\usetikzlibrary{positioning}
\graphicspath{ {./images/} }
\hypersetup{
  colorlinks=true,
  linktoc=all,
  linkcolor=black,
}
\setcounter{tocdepth}{3}
\setcounter{secnumdepth}{3}

\newcommand{\subtitle}[1]{%
  \posttitle{%
    \par\end{center}
    \begin{center}\large#1\end{center}
    \vskip0.5em}%
}

\title{Projekt programistyczny}
\subtitle{
  Podstawy internetu rzeczy \\
  Prowadzący laboratorium: dr inż. Krzysztof Chudzik
}
\author{
  Kacper Gaudyn, nr indeksu 266873 \\
  Kuba Krąpiec, nr indeksu 266503 \\
  Michał Pesta, nr indeksu 266899 \\
  Michał Trojanowski, nr indeksu 266864
}
\date{Data ukończenia pracy: ...}

\begin{document}
  \begin{titlingpage}
    \maketitle
  \end{titlingpage}
  \tableofcontents

  \newpage
  \section{Wymagania projektowe}
\subsection{Podstawowe wymagania funkcjonalne}
\subsection{Podstawowe wymagania niefunkcjonalne}

  \section{Opis architektury systemu}
\subsection{Elementy architektury z opisem}
\begin{itemize}
  \item{Baza danych - zawiera wszystkie dane}
  \item{REST API - zarządza danymi bazy danych i udostępnia odpowiednie operacje innym podmiotom}
  \item{Panel administracyjny - służy administratorom do zarządzania danymi}
  \item{Broker MQTT - komunikuje się z terminalami w pojazdach i przekazuje informacje o przejechanych przystankach i płatnościach}
  \item{Terminal - znajduje się w każdym pojeździe, umożliwia kierowcy ustalanie trasy i zmianę przystanków, oraz umożliwia pracownikom płacenie za swoje przejazdy}
  \item{
    Użytkownicy
    \begin{itemize}
      \item{Kierowca - kieruje pojazdem i na terminalu może zmieniać trasy, przystanki}
      \item{Pracownik - może wsiadać do pojazdów i płacić w terminalu za przejazd kartą}
      \item{Administrator - zarządza danymi w systemie}
    \end{itemize}
  }
\end{itemize}
\subsection{Graficzna reprezentacja architektury}
\begin{figure}[H]
  \centering
  \begin{tikzpicture}[>=Latex,
    user/.style={draw=gray,dashed}
  ]
    \node at (0,0) [rectangle,draw] (database) {Baza danych};
    \node [below=of database,rectangle,draw] (rest) {REST API};
    \node [below left=of rest,rectangle,draw] (frontend_admin) {Panel administracyjny};
    \node [right=of database,rectangle,draw] (broker) {Broker MQTT};
    \node [below=of broker,rectangle,draw] (terminal) {Terminal};

    \node[below=0.5 of frontend_admin,rectangle,draw,user] (admin) {Administrator};
    \node[below left=0.5 of terminal,rectangle,draw,user] (driver) {Kierowca};
    \node[below right=0.5 of terminal,rectangle,draw,user] (worker) {Pracownik};

    \draw [<->] (database) -- (rest);
    \draw [<->] (rest) -- (frontend_admin);
    \draw [<->] (database) -- (broker);
    \draw [<->] (broker) -- (terminal);
    \draw [->] (rest) -- (terminal);

    \draw [<->] (admin) -- (frontend_admin);
    \draw [<->] (driver) -- (terminal);
    \draw [<->] (worker) -- (terminal);
  \end{tikzpicture}
  \caption{Diagram elementów architektury z kierunkami przekazywania danych}
\end{figure}
\subsection{Baza danych}
\subsubsection{Schemat bazy danych}
\begin{figure}[H]
  \centering
  \begin{tikzpicture}[>=Latex,
    db/.style={
      draw, matrix of nodes,
      nodes={
        node family/text width/.expanded=%
          \tikzmatrixname-\the\pgfmatrixcurrentcolumn,
        node family/text width align=left,
        inner xsep=+.5\tabcolsep, inner ysep=+0pt, align=left},
      inner sep=.5\pgflinewidth,
      font=\strut\ttfamily,
    }
  ]
    \matrix[db,label=Courses,row 1/.style={nodes={fill=red!20}}] (courses) {
      CourseID & int & \node (courses_courseid) {PK}; \\
      CourseName & varchar(255) \\
    };
    \matrix[db,label=Stops,below=of courses,row 1/.style={nodes={fill=red!20}}] (stops) {
      StopID & int & \node (stops_stopid) {PK}; \\
      StopName & varchar(255) \\
    };
    \matrix[db,label=Assignments,below=of stops,row 1/.style={nodes={fill=yellow!20}},row 2/.style={nodes={fill=yellow!20}}] (assignments) {
      CourseID & int & \node (assignments_courseid) {PK, FK}; \\
      StopID & int & \node (assignments_stopid) {PK, FK}; \\
      StopNumber & int & \node (assignments_stopnumber) {}; \\
    };
    \matrix[db,label=Buses,below=of assignments,row 1/.style={nodes={fill=red!20}},row 2/.style={nodes={fill=blue!20}},row 3/.style={nodes={fill=blue!20}}] (assignments) {
      BusID & int & \node (buses_busid) {PK}; \\
      CourseID & int & \node (buses_courseid) {FK}; \\
      StopNumber & int & \node (buses_stopnumber) {FK}; \\
    };
    \matrix[db,label=Workers,below=of assignments,row 1/.style={nodes={fill=red!20}}] (workers) {
      WorkerID & int & \node (workers_workerid) {PK}; \\
      WorkerFirstName & varchar(255) \\
      WorkerLastName & varchar(255) \\
      WorkerBalance & int \\
      WorkerCardID & varchar(255) & \node(workers_workercardid) {}; \\
    };
    \matrix[db,label=CurrentRides,below=of workers,row 1/.style={nodes={fill=red!20}},row 2/.style={nodes={fill=blue!20}},row 3/.style={nodes={fill=blue!20}}] {
      RideID & int & PK \\
      WorkerCardID & varchar(255) & \node (currentrides_workercardid) {FK}; \\
      BusID & int & \node(currentrides_busid) {FK}; \\
      StopsTraveled & int \\
    };
    \draw [<-] (courses_courseid.east) -- ++(0.5,0) |- (assignments_courseid.east);
    \draw [<-] (stops_stopid.east) -- ++(0.5,0) |- (assignments_stopid.east);
    \draw [<-] (workers_workercardid.east) -- ++(0.5,0) |- (currentrides_workercardid.east);
    \draw [<-] (courses_courseid.east) -- ++(1,0) |- (buses_courseid.east);
    \draw [<-] (assignments_stopnumber.east) -- ++(0.5,0) |- (buses_stopnumber.east);
    \draw [<-] (buses_busid.east) -- ++(2,0) |- (currentrides_busid.east);
  \end{tikzpicture}
  \caption{Tabele bazy danych z zaznaczonymi relacjami między nimi}
\end{figure}
\subsubsection{Scenariusze i ich wpływ na dane}
\begin{enumerate}
  \item{
    Pracownik wsiada na przystanku, przykłada kartę i po przejechaniu 3 przystanków przykłada ją znowu aby zapłacić i wysiada. \\
    Zakładamy, że każdy przystanek kosztuje 1, czyli pracownik zapłaci 3.
    \begin{figure}[H]
      \centering
      \begin{tikzpicture}[>=Latex,
        db/.style={
          draw, matrix of nodes,
          nodes={
            node family/text width/.expanded=%
              \tikzmatrixname-\the\pgfmatrixcurrentcolumn,
            node family/text width align=left,
            inner xsep=+.5\tabcolsep, inner ysep=+0pt, align=left},
          inner sep=.5\pgflinewidth,
          font=\strut\ttfamily,
          row 1/.style={nodes={fill=gray!20}}
        }
      ]
        \matrix[db,label=Workers] (workers) {
          WorkerID & WorkerFirstName & WorkerLastName & WorkerBalance & WorkerCardID \\
          1 & Jan & Kowalski & 100 & ABCD \\
        };
      \end{tikzpicture}
      \caption{Dane w bazie przed wykonaniem scenariusza}
    \end{figure}
    \begin{figure}[H]
      \centering
      \begin{tikzpicture}[>=Latex,
        db/.style={
          draw, matrix of nodes,
          nodes={
            node family/text width/.expanded=%
              \tikzmatrixname-\the\pgfmatrixcurrentcolumn,
            node family/text width align=left,
            inner xsep=+.5\tabcolsep, inner ysep=+0pt, align=left},
          inner sep=.5\pgflinewidth,
          font=\strut\ttfamily,
          row 1/.style={nodes={fill=gray!20}}
        }
      ]
        \matrix[db,label=Workers] (workers) {
          WorkerID & WorkerFirstName & WorkerLastName & WorkerBalance & WorkerCardID \\
          1 & Jan & Kowalski & 97 & ABCD \\
        };
        \matrix[db,label=CurrentRides,below=of workers] {
          RideID & WorkerCardID & BusID & StopsTraveled \\
          \vdots \\
          5 & ABCD & 1 & 3 \\
          \vdots \\
        };
      \end{tikzpicture}
      \caption{Dane w bazie po wykonaniu scenariusza}
    \end{figure}
    W tabeli \verb|Workers| zmieniła się kolumna \verb|WorkerBalance|, a w tabeli \verb|CurrentRides| został dodany nowy rekord.
  }
  \item{
    Kierowca pojazdu który nie jest na żadnej trasie, ustala nową trasę o nazwie \verb|MediumLengthCourse|.
    \begin{figure}[H]
      \centering
      \begin{tikzpicture}[>=Latex,
        db/.style={
          draw, matrix of nodes,
          nodes={
            node family/text width/.expanded=%
              \tikzmatrixname-\the\pgfmatrixcurrentcolumn,
            node family/text width align=left,
            inner xsep=+.5\tabcolsep, inner ysep=+0pt, align=left},
          inner sep=.5\pgflinewidth,
          font=\strut\ttfamily,
          row 1/.style={nodes={fill=gray!20}}
        }
      ]
        \matrix[db,label=Courses] (courses) {
          CourseID & CourseName \\
          \vdots \\
          2 & MediumLengthCourse \\
          \vdots \\
        };
        \matrix[db,label=Buses,below=of courses] (buses) {
          BusID & CourseID & StopNumber \\
          1 & NULL & NULL \\
        };
      \end{tikzpicture}
      \caption{Dane w bazie przed wykonaniem scenariusza}
    \end{figure}
    \begin{figure}[H]
      \centering
      \begin{tikzpicture}[>=Latex,
        db/.style={
          draw, matrix of nodes,
          nodes={
            node family/text width/.expanded=%
              \tikzmatrixname-\the\pgfmatrixcurrentcolumn,
            node family/text width align=left,
            inner xsep=+.5\tabcolsep, inner ysep=+0pt, align=left},
          inner sep=.5\pgflinewidth,
          font=\strut\ttfamily,
          row 1/.style={nodes={fill=gray!20}}
        }
      ]
        \matrix[db,label=Buses] (buses) {
          BusID & CourseID & StopNumber \\
          1 & 2 & 1 \\
        };
      \end{tikzpicture}
      \caption{Dane w bazie po wykonaniu scenariusza}
    \end{figure}
  }
  \item{
    Kierowca pojazdu który jest na trasie \verb|ShortCourse|, zmienia przystanek z \verb|Our Company| na \verb|Amusement Park|.
    \begin{figure}[H]
      \centering
      \begin{tikzpicture}[>=Latex,
        db/.style={
          draw, matrix of nodes,
          nodes={
            node family/text width/.expanded=%
              \tikzmatrixname-\the\pgfmatrixcurrentcolumn,
            node family/text width align=left,
            inner xsep=+.5\tabcolsep, inner ysep=+0pt, align=left},
          inner sep=.5\pgflinewidth,
          font=\strut\ttfamily,
          row 1/.style={nodes={fill=gray!20}}
        }
      ]
        \matrix[db,label=Courses] (courses) {
          CourseID & CourseName \\
          \vdots \\
          3 & ShortCourse \\
          \vdots \\
        };
        \matrix[db,label=Stops,below=of courses] (stops) {
          StopID & StopName \\
          \vdots \\
          12 & Our Company \\
          \vdots \\
          14 & Amusement Park \\
          \vdots \\
        };
        \matrix[db,label=Assignments,below=of stops] (assignments) {
          CourseID & StopID & StopNumber \\
          \vdots \\
          3 & 12 & 1 \\
          3 & 14 & 2 \\
          \vdots \\
        };
        \matrix[db,label=Buses,below=of assignments] (buses) {
          BusID & CourseID & StopNumber \\
          1 & 3 & 1 \\
        };
      \end{tikzpicture}
      \caption{Dane w bazie przed wykonaniem scenariusza}
    \end{figure}
    \begin{figure}[H]
      \centering
      \begin{tikzpicture}[>=Latex,
        db/.style={
          draw, matrix of nodes,
          nodes={
            node family/text width/.expanded=%
              \tikzmatrixname-\the\pgfmatrixcurrentcolumn,
            node family/text width align=left,
            inner xsep=+.5\tabcolsep, inner ysep=+0pt, align=left},
          inner sep=.5\pgflinewidth,
          font=\strut\ttfamily,
          row 1/.style={nodes={fill=gray!20}}
        }
      ]
        \matrix[db,label=Buses] (buses) {
          BusID & CourseID & StopNumber \\
          1 & 3 & 2 \\
        };
      \end{tikzpicture}
      \caption{Dane w bazie po wykonaniu scenariusza}
    \end{figure}
  }
\end{enumerate}

  \section{Opis implementacji i zastosowanych rozwiązań}
\subsection{Wykorzystane języki, technologie}
\begin{itemize}
  \item{Baza danych - SQLite}
  \item{REST API - Python, FastAPI}
  \item{Panel administracyjny - JavaScript, Svelte}
  \item{Broker MQTT - Python}
  \item{Terminal - Python}
\end{itemize}
\subsection{Najważniejsze funkcje systemu}
\subsubsection{Ustalanie trasy przez kierowcę}
\begin{lstlisting}[language={Python}, caption={Terminal, Lokalizacja: \texttt{client/client.py}}]
def begin_route() -> None:
  global stop_rfid_polling
  global current_stop_index
  global routes
  global route_index
  current_stop_index = 0
  GPIO.add_event_detect(buttonGreen, GPIO.FALLING, callback=on_green_button_while_on_route, bouncetime=500)
  GPIO.add_event_detect(buttonRed, GPIO.FALLING, callback=on_red_button_while_on_route, bouncetime=500)
  client.publish("buses/driver", f"choose_course?bus={bus_id}&course={routes[route_index].name}")
  draw_stops_screen(routes[route_index], current_stop_index)
  stop_rfid_polling = False
  rfid_thread = threading.Thread(target=listen_rfid)
  rfid_thread.daemon = True
  rfid_thread.start()
\end{lstlisting}
Po wybraniu trasy przez kierowcę terminal komunikuje się z brokerem MQTT poprzez temat \verb|buses/driver| i wysyła mu odpowiednie dane, następnie też aktywowana jest możliwość skanowania karty pracownika zaimplementowana w wątku \verb|rfid_thread|. Dodatkowo funkcjonalność przycisków: zielonego i czerwonego zostaje zmieniona, tak aby była odpowiedzialna już za zmianę przystanków, a wyświetlacz pokazuje przystanki.
\begin{lstlisting}[language={Python}, caption={Broker MQTT, Lokalizacja: \texttt{backend/mqtt\_server.py}}]
def choose_course(bus_id: int, course_name: str):
  course_id = db_management.select('Courses', ['CourseID'], [('CourseName', course_name)])[0][0]
  db_management.update('Buses', ('CourseID', course_id), ('BusID', bus_id))
  db_management.update('Buses', ('StopNumber', 1), ('BusID', bus_id))
\end{lstlisting}
Broker MQTT po otrzymaniu danych od terminala, aktualizuje bazę danych.
\subsubsection{Zmiana przystanku przez kierowcę}
\begin{lstlisting}[language={Python}, caption={Terminal, Lokalizacja: \texttt{client/client.py}}]
def on_green_button_while_on_route(_) -> None:
  global current_stop_index
  global routes
  global route_index
  global stop_rfid_polling

  route = routes[route_index]
  client.publish("buses/driver", f"next_stop?bus={bus_id}")
  if current_stop_index < len(route.stops) - 1:
      current_stop_index += 1
      draw_stops_screen(route, current_stop_index)
  else:
      stop_rfid_polling = True
      GPIO.remove_event_detect(buttonGreen)
      GPIO.remove_event_detect(buttonRed)
      route_index = 0
      current_stop_index = 0
      select_route()
\end{lstlisting}
Przy zmianie przystanku przez kierowcę terminal wysyła dane do brokera MQTT, następnie informacja o aktualnym przystanku na wyświetlaczu zostaje zmieniona. Jeżeli następnego przystanku nie ma, to wtedy trasa jest automatycznie zakończona.
\begin{lstlisting}[language={Python}, caption={Broker MQTT, Lokalizacja: \texttt{backend/mqtt\_server.py}}]
def next_stop(bus_id: int):
  bus_data = db_management.select('Buses', ['BusID', 'CourseID', 'StopNumber'],
                                  [('BusID', bus_id)])
  if not bus_data:
      return NO_SUCH_BUS
  bus_id, course_id, stop_number = bus_data[0]
  try:
      stops = [tup[0] for tup in db_management.select('Assignments', ['StopID'],
                                                      [('CourseID', course_id)])]
  except OperationalError:
      return BUS_NOT_ON_ROUTE
  new_stop_number = stop_number + 1
  if new_stop_number == 0 or new_stop_number > len(stops):
      for attribute_name in ['CourseID', 'StopNumber']:
          db_management.update('Buses', (attribute_name, 'null'), ('BusID', bus_id))
      return ROUTE_ENDED
  else:
      db_management.update('Buses', ('stopNumber', new_stop_number), ('BusID', bus_id))
      new_stop_id = db_management.select('Assignments', ['StopID'],
                                         [('CourseID', course_id), ('stopNumber', new_stop_number)])[0][0]
      new_stop_name = db_management.select('Stops', ['StopName'], [('StopID', new_stop_id)])[0][0]
      add_stop_to_workers(bus_id)
      return f'{new_stop_number}-{new_stop_name}'
\end{lstlisting}
Broker MQTT po otrzymaniu danych od terminala aktualizuje przystanek w bazie danych, jeżeli podany pojazd jest na trasie, następnie aktualizowana jest liczba przejechanych przystanków dla każdego pracownika który znajduje się na trasie pojazdu.
\subsubsection{Płacenie za przejazd przez pracownika}
\begin{lstlisting}[language={Python}, caption={Terminal, Lokalizacja: \texttt{client/client.py}}]
def on_card_scanned(uid: list[int]) -> None:
  global last_card_scan_value_time
  global current_stop_index
  global routes
  global route_index
  uid_int = int(''.join(list(map(lambda e: str(e), uid))))
  print(f'scanned {uid_int}')
  (last_value, last_time) = last_card_scan_value_time
  if last_time is not None and last_time + datetime.timedelta(seconds=7) > datetime.datetime.now():
      return
  last_card_scan_value_time = (uid_int, datetime.datetime.now())
  client.publish("buses/worker", f"use_card?card={uid_int}&bus={bus_id}")
\end{lstlisting}
Po przyłożeniu karty przez pracownika, terminal wysyła dane do brokera MQTT.
\pagebreak
\begin{lstlisting}[language={Python}, caption={Broker MQTT, Lokalizacja: \texttt{backend/mqtt\_server.py}}]
def card_used(card_id: str, bus_id: int):
  riding_workers = [tup[0] for tup in db_management.select_all('CurrentRides', ['WorkerCardID'])]
  if int(card_id) in riding_workers:
      return worker_gets_out(card_id)
  else:
      return worker_gets_in(card_id, bus_id)


def worker_gets_in(card_id: str, bus_id: int):
  if int(card_id) not in [tup[0] for tup in db_management.select_all('Workers', ['WorkerCardID'])]:
      return 'Invalid card.'
  try:
      max_ride_id = max([tup[0] for tup in db_management.select_all('CurrentRides', ['RideID'])])
  except ValueError:
      max_ride_id = 0
  db_management.insert('CurrentRides', (max_ride_id + 1, card_id, bus_id, 0))
  return f'Card validated succesfully.'


def worker_gets_out(card_id: str):
  stops_traveled = db_management.select('CurrentRides', ['StopsTraveled'], [('WorkerCardID', card_id)])[0][0]
  db_management.delete('CurrentRides', ('WorkerCardID', card_id))
  current_balance = db_management.select('Workers', ['WorkerBalance'], [('WorkerCardID', card_id)])[0][0]
  db_management.update('Workers', ('WorkerBalance', current_balance - stops_traveled), ('WorkerCardID', card_id))
  return f'Balance after ride: {current_balance - stops_traveled}.'
\end{lstlisting}
Broker MQTT po otrzymaniu danych od terminala, sprawdza czy pracownik zaczyna, lub kończy przejazd:
\begin{itemize}
  \item{jeżeli zaczyna przejazd, to po sprawdzeniu czy taki pracownik istnieje w bazie, zostaje dodany rekord o aktualnym przejeździe,}
  \item{jeżeli kończy przejazd, to z jego konta zostaje potrącona kwota zależna od przejechanych przystanków, oraz jego przejazd zostaje usunięty z bazy.}
\end{itemize}
\subsubsection{Dodanie pieniędzy do konta pracownika przez administratora}
\begin{lstlisting}[language={JavaScript}, caption={Panel administracyjny, Lokalizacja: \texttt{frontend/app/src/routes/employees/+page.svelte}}]
async function editBalance(){
    if(bonusData.bonus == null) bonusData.bonus = 0.0
    try{
        const url = `/addbalance/${bonusData.worker_id}?value=${bonusData.bonus}`
        const response = await fetch(baseUrl + url, {
            method: "GET",
        })
        if(response.ok){
            console.log("Form data sent successfully");
            employees = await getEmployeesData();
        }else{
            console.error("error sending form data", response.status);
        }
    }catch(e){
        console.log(e);
    }
}
\end{lstlisting}
Panel administracyjny przekazuje odpowiednie dane do REST API, następnie pobiera zaktualizowane dane o pracownikach.
\begin{lstlisting}[language={Python}, caption={REST API, Lokalizacja: \texttt{backend/server.py}}]
@app.get("/addbalance/{worker_id}")
async def add_balance_endpoint(worker_id: int, value: float):
    current_balance = db_management.select('Workers', ['WorkerBalance'], [('WorkerID', worker_id)])[0][0]
    db_management.update('Workers', ('WorkerBalance', current_balance + value), ('WorkerID', worker_id))
    return {'success': f'Balance of worker with ID {worker_id} has changed from {current_balance} to '
                        f'{current_balance + value}'}
\end{lstlisting}
REST API po otrzymaniu danych, aktualizuje bazę danych.
\subsection{Implementacja MQTT}
\subsubsection{Broker MQTT}
\begin{lstlisting}[language={Python}, caption={Broker MQTT, Lokalizacja: \texttt{backend/mqtt\_server.py}}]
import paho.mqtt.client as mqtt
(...)

broker_ip = "0.0.0.0"

CLIENT = mqtt.Client()

(...)


def process_message(client, userdata, message):
    global CLIENT
    message_decoded = (str(message.payload.decode("utf-8"))).split(".")[0]
    print(message_decoded)
    message_dict = query_string_to_dict(message_decoded)
    if 'next_stop' in message_dict:
        next_stop(message_dict['next_stop']['bus'])
    elif 'choose_course' in message_dict:
        choose_course(message_dict['choose_course']['bus'], message_dict['choose_course']['course'])
    elif 'use_card' in message_dict:
        result_code = card_used(message_dict['use_card']['card'], message_dict['use_card']['bus'])
        CLIENT.publish('response/success', result_code)


def connect_to_broker():
    CLIENT.connect(broker_ip)
    CLIENT.on_message = process_message
    CLIENT.loop_start()
    CLIENT.subscribe("buses/#")


def disconnect_from_broker():
    CLIENT.loop_stop()
    CLIENT.disconnect()


def run_mqtt_server():
    connect_to_broker()
    input()


(...)


if __name__ == "__main__":
    run_mqtt_server()
\end{lstlisting}
Broker MQTT subskrybuje każdy temat, który zaczyna się od \verb|buses/|, każda wiadomość która jest odbierana przez broker jest w formacie \verb|query string|, natomiast odpowiedzi są wysyłane tematem \verb|response/success|.
\pagebreak
\subsubsection{Terminal}
\begin{lstlisting}[language={Python}, caption={Terminal, Lokalizacja: \texttt{client/client.py}}]
#!/usr/bin/env python3

import paho.mqtt.client as mqtt
(...)

server_ip = "10.108.33.123"
bus_id = 0

client = mqtt.Client()
(...)


def begin_route() -> None:
    (...)
    client.publish("buses/driver", f"choose_course?bus={bus_id}&course={routes[route_index].name}")
    (...)


def on_mqtt_message(client, userdata, message):
    message_decoded = str(message.payload.decode('utf-8'))
    print(f'message received: {message_decoded}')
    draw_message(message_decoded)
    timer = threading.Timer(7, draw_stops_screen, args=(routes[route_index], current_stop_index))
    timer.start()


(...)


if __name__ == "__main__":
    try:
        disp.Init()
        disp.clear()

        client.connect(server_ip)
        client.on_message = on_mqtt_message
        client.loop_start()
        client.subscribe("response/#")

        (...)

        while True:
            _ = input()
    (...)
\end{lstlisting}
Terminal subskrybuje wszystkie tematy zaczynające się od \verb|response/|, oraz publikuje dane na różnych tematach zaczynających się od \verb|buses/|. Dane które są odebrane z \verb|response/| są przekazywane do wyświetlacza, który je pokazuje.
\subsection{Implementacja terminala}
\subsubsection{Pobieranie danych o kursach i ich przystankach}
\begin{lstlisting}[language={Python}, caption={Terminal, Lokalizacja: \texttt{client/client.py}}]
def fetch_routes() -> list[Route]:
  with urllib.request.urlopen(f'http://{server_ip}:55555/courses') as url:
      data = json.load(url)
      for route_name in data:
          stops = map(lambda stop_name: Stop(stop_name), data[route_name])
          routes.append(Route(route_name, list(stops)))
  return routes
\end{lstlisting}
Do pobierania danych o kursach i przystankach terminal wykorzystuje REST API.
\pagebreak
\begin{lstlisting}[language={Python}, caption={REST API, Lokalizacja: \texttt{backend/server.py}}]
@app.get("/courses")
async def courses_endpoint():
    result = {}
    courses = db_management.select_all('Courses', ['CourseID', 'courseName'])
    for course_id, course_name in courses:
        stops = db_management.select('Assignments', ['StopID', 'StopNumber'], [('CourseID', course_id)])
        result[course_name] = ['' for _ in range(len(stops))]
        for stop_id, stop_number in stops:
            stop_name = db_management.select('Stops', ['StopName'], [('StopID', stop_id)])[0][0]
            result[course_name][stop_number-1] = stop_name
    return result
\end{lstlisting}
REST API pobiera z bazy danych kursy, następnie pobiera ich przystanki i je przypisuje do zwracanych danych.

  \section{Opis działania i prezentacja interfejsu}
\subsection{Sposób instalacji i uruchomienia aplikacji}
\subsubsection{Zawartość plików}
\begin{itemize}
  \item{
    \verb|backend/|
    \begin{itemize}
      \item{baza danych - pliki: \verb|db_management.py|, \verb|init.sql|, \verb|insert_example.sql|}
      \item{REST API - plik: \verb|server.py|}
      \item{broker MQTT - pliki: \verb|mqtt_server.py|, \verb|mqtt_client_TEST.py|}
    \end{itemize}
  }
  \item{\verb|frontend/app| - panel administracyjny}
  \item{\verb|client/| - terminal}
\end{itemize}
\subsubsection{Instalacja}
\begin{enumerate}
  \item{
    Baza danych
    \begin{enumerate}
      \item{przejdź do folderu \verb|backend|}
      \item{uruchom plik \verb|db_management.py|}
      \item{w folderze powinien utworzyć się plik \verb|alpha.db|, zawierający przykładowe dane}
    \end{enumerate}
  }
  \item{
    REST API i broker MQTT
    \begin{enumerate}
      \item{zainstaluj wymagane pakiety za pomocą komendy \verb|pip install -r requirements.txt|}
      \item{
        upewnij się, że masz zainstalowane i uruchomione \href{https://mosquitto.org/download/}{Mosquitto} \\
        UWAGA: w pliku konfiguracyjnym Mosquitto powinny się znaleźć następujące linijki, aby urządzenia mogły się połączyć:
        \begin{lstlisting}[caption={Dodatek do pliku \texttt{mosquitto.conf}}]
        allow_anonymous true
        listener 1883 0.0.0.0
        \end{lstlisting}
      }
      \item{uruchom plik \verb|server.py|}
      \item{działanie REST API można testować pod adresem: \url{http://localhost:55555/docs}}
    \end{enumerate}
  }
  \item{
    Panel administracyjny
    \begin{enumerate}
      \item{upewnij się, że masz zainstalowany \href{https://nodejs.org/en}{Node.js}}
      \item{przejdź do folderu \verb|frontend/app|}
      \item{zainstaluj pakiety za pomocą komendy \verb|npm install|}
      \item{wprowadź komendę \verb|npm run dev|}
      \item{działanie można testować pod adresem: \url{http://localhost:5173/}}
    \end{enumerate}
  }
  \item{
    Terminal (wersja Raspberry Pi)
    \begin{enumerate}
      \item{przejdź do folderu \verb|client|}
      \item{otwórz plik \verb|client.py| i zmodyfikuj zmienną \verb|server_ip| tak, aby odpowiadała adresowi IP brokera MQTT}
      \item{uruchom plik \verb|client.py|}
    \end{enumerate}
  }
  \item{
    Terminal (wersja testowa)
    \begin{enumerate}
      \item{przejdź do folderu \verb|backend|}
      \item{otwórz plik \verb|mqtt_client_TEST.py| i zmodyfikuj zmienną \verb|BROKER_IP| tak, aby odpowiadała adresowi IP brokera MQTT}
      \item{uruchom plik \verb|mqtt_client_TEST.py|}
    \end{enumerate}
  }
\end{enumerate}
\subsection{Przedstawienie działania aplikacji}
\subsubsection{Panel administracyjny}
\begin{figure}[H]
  \centering
  \includegraphics[width=1\textwidth]{admin-panel/employees}
  \caption{Widok pracowników w panelu administracyjnym}
\end{figure}
Przykładowo dodamy nowego pracownika i dodamy mu pieniądze do stanu konta, wypełniając odpowiednie pola i klikając znak \verb|+|.
\begin{figure}[H]
  \centering
  \includegraphics[width=1\textwidth]{admin-panel/employees-add}
  \caption{Dodany nowy pracownik}
\end{figure}
\begin{figure}[H]
  \centering
  \includegraphics[width=0.75\textwidth]{admin-panel/routes-course}
  \caption{Widok przejazdów w panelu administracyjnym, kursy}
\end{figure}
\begin{figure}[H]
  \centering
  \includegraphics[width=0.5\textwidth]{admin-panel/routes-stops}
  \caption{Widok przejazdów w panelu administracyjnym, przystanki}
\end{figure}
\begin{figure}[H]
  \centering
  \includegraphics[width=0.5\textwidth]{admin-panel/routes-add}
  \caption{Widok przejazdów w panelu administracyjnym, dodawanie nowych danych}
\end{figure}
Przykładowo dodamy nowy przystanek o nazwie \verb|Test Stop|, oraz kurs o nazwie \verb|TestCourse|, który będzie go zawierał.
\begin{figure}[H]
  \centering
  \includegraphics[width=0.75\textwidth]{admin-panel/routes-new}
  \caption{Dodany nowy kurs zawierający nowy przystanek}
\end{figure}
\begin{figure}[H]
  \centering
  \includegraphics[width=1\textwidth]{admin-panel/map}
  \caption{Mapa pojazdów odbywających kursy}
\end{figure}
W terminalu zmieniamy przystanek na którym pojazd się znajduje.
\begin{figure}[H]
  \centering
  \includegraphics[width=1\textwidth]{admin-panel/map-nextstop}
  \caption{Zmiana przystanku widoczna na mapie}
\end{figure}
\subsubsection{Terminal (wersja Raspberry Pi)}
\begin{figure}[H]
  \centering
  \includegraphics[width=0.4\textwidth]{terminal-pi/course}
  \caption{Widok wyboru trasy}
\end{figure}
\begin{figure}[H]
  \centering
  \includegraphics[width=0.4\textwidth]{terminal-pi/stop}
  \caption{Widok aktualnego przystanku na trasie}
\end{figure}
\subsubsection{Terminal (wersja testowa)}
\begin{figure}[H]
  \centering
  \includegraphics[width=0.4\textwidth]{terminal-test/menu}
  \caption{Widok menu}
\end{figure}
\begin{figure}[H]
  \centering
  \includegraphics[width=0.6\textwidth]{terminal-test/example}
  \caption{Przykład wyboru kursu, zmiany przystanków i płacenia za przejazd przez pracownika}
\end{figure}

  \section{Wkład pracy Autorów}
\subsection{Pierwszy tydzień (8.01 - 14.01)}
\begin{itemize}
  \item{
    Kacper Gaudyn
    \begin{itemize}
      \item{struktura bazy danych, tabele: \verb|Courses|, \verb|Stops|, \verb|Assignments|, \verb|Workers|, \verb|CurrentRides|}
      \item{tworzenie pliku bazy danych, podstawowe operacje na danych}
    \end{itemize}
  }
  \item{
    Jakub Krąpiec
    \begin{itemize}
      \item{terminal - obsługa przycisków, struktura przystanków i tras}
      \item{dodanie możliwości wyboru trasy i przystanków za pomocą enkodera, obsługa przycisków rozbudowana o nowe funkcjonalności, obsługa RFID}
    \end{itemize}
  }
  \item{
    Michał Pesta
    \begin{itemize}
      \item{utworzenie szkieletu panelu administracyjnego, dodanie paska nawigacji}
      \item{dodanie strony z informacjami o pracownikach}
    \end{itemize}
  }
  \item{
    Michał Trojanowski
    \begin{itemize}
      \item{utworzenie szkieletu raportu, dodanie do raportu wymagań projektowych i opisu architektury systemu}
      \item{terminal - wyświetlanie informacji na wyświetlaczu}
    \end{itemize}
  }
\end{itemize}
\subsection{Drugi tydzień (15.01 - 21.01)}
\begin{itemize}
  \item{
    Kacper Gaudyn
    \begin{itemize}
      \item{dodanie szkieletu REST API}
      \item{dodanie do bazy danych tabeli \verb|Buses|, pobieranie pracowników i pojazdów w REST API, drobne poprawki}
      \item{dodanie brokera MQTT, dokumentacji do brokera MQTT oraz terminala w wersji testowej}
    \end{itemize}
  }
  \item{
    Jakub Krąpiec
    \begin{itemize}
      \item{terminal - poprawki w obsłudze przycisków i enkodera}
      \item{poprawki w strukturze bazy danych, dodano pobieranie kursów w REST API}
      \item{terminal - zmiany wizualne w wyświetlaniu informacji}
    \end{itemize}
  }
  \item{
    Michał Pesta
    \begin{itemize}
      \item{dodanie przykładowych danych do bazy danych, dodana funkcjonalność zmiany przystanków w REST API}
      \item{pobieranie danych z REST API w panelu administratora, dodanie podglądu pojazdów w ruchu}
      \item{dodanie dokumentacji do REST API, drobne poprawki w bazie danych i REST API, dodanie funkcjonalności związanych z pracownikami w REST API}
    \end{itemize}
  }
  \item{
    Michał Trojanowski
    \begin{itemize}
      \item{aktualizacja architektury systemu w raporcie, dodanie do raportu częściowego opisu implementacji, dodanie do raportu wstępu}
      \item{terminal - dodana komunikacja z brokerem MQTT}
      \item{terminal - wyświetlanie informacji od brokera}
    \end{itemize}
  }
\end{itemize}
\subsection{Trzeci tydzień (22.01 - 26.01)}
\begin{itemize}
  \item{
    Kacper Gaudyn
    \begin{itemize}
      \item{poprawki w testowych danych bazy danych, poprawki w obsłudze identyfikatora karty pracownika w brokerze MQTT}
      \item{drobne poprawki w REST API i dokumentacji}
      \item{dokończenie opisu architektury w raporcie}
    \end{itemize}
  }
  \item{
    Jakub Krąpiec
    \begin{itemize}
      \item{drobne poprawki w raporcie, zmiany w komunikacji z brokerem MQTT w terminalu, pobieranie danych o trasach i przystankach z REST API w terminalu}
      \item{drobne poprawki w terminalu}
      \item{dodanie opisu implementacji w raporcie}
    \end{itemize}
  }
  \item{
    Michał Pesta
    \begin{itemize}
      \item{obsługa przystanków w panelu administracyjnym; komunikacja z REST API, poprawki w ustawieniach CORS}
      \item{drobne poprawki w REST API oraz panelu administracyjnym}
      \item{dodanie opisu działania w raporcie}
    \end{itemize}
  }
  \item{
    Michał Trojanowski
    \begin{itemize}
      \item{obsługa kursów w panelu administracyjnym}
      \item{drobne poprawki w REST API}
      \item{dodanie do raportu: wkładu pracy i podsumowania, zrobienie prezentacji}
    \end{itemize}
  }
\end{itemize}

  \section{Podsumowanie}
\subsection{Stopień zgodności z wymaganiami}
\subsubsection{Wymagania funkcjonalne}
\begin{enumerate}
  \item{\hyperref[func_1]{\textcolor{green}{Zrealizowane} - odpowiedni przycisk w terminalu po wybraniu trasy}}
  \item{\hyperref[func_2]{\textcolor{green}{Zrealizowane} - wybór za pomocą enkodera}}
  \item{\hyperref[func_3]{\textcolor{green}{Zrealizowane} - realizowane poprzez przyciski}}
  \item{\hyperref[func_4]{\textcolor{green}{Zrealizowane}}}
  \item{\hyperref[func_5]{\textcolor{green}{Zrealizowane} - należy przyłożyć kartę przy wsiadaniu i wysiadaniu}}
  \item{\hyperref[func_6]{\textcolor{green}{Zrealizowane} - koszt wynosi 1}}
  \item{\hyperref[func_7]{\textcolor{green}{Zrealizowane} - panel administratora}}
  \item{\hyperref[func_8]{\textcolor{green}{Zrealizowane} - panel administratora}}
  \item{\hyperref[func_9]{\textcolor{green}{Zrealizowane} - panel administratora}}
\end{enumerate}
\subsubsection{Wymagania niefunkcjonalne}
\begin{enumerate}
  \item{\hyperref[nfunc_1]{\textcolor{green}{Zrealizowane}}}
  \item{\hyperref[nfunc_2]{\textcolor{green}{Zrealizowane}}}
\end{enumerate}
\subsection{Napotkane problemy w implementacji}
\subsubsection{Uruchomienie systemu na komputerze laboratoryjnym}
Na komputerach laboratoryjnych nie ma zainstalowanego \verb|Node.js|, a użytkownik studenta nie ma uprawnień administratora, więc też nie było możliwości instalacji. Problem dotyczył panelu administracyjnego i został rozwiązany poprzez zmianę środowiska na Raspberry Pi, tam już instalacja była możliwa.

  \include{literatura.tex}
\end{document}
