\section{Opis implementacji i zastosowanych rozwiązań}
\subsection{Wykorzystane języki, technologie}
\begin{itemize}
  \item{Baza danych - SQLite}
  \item{REST API - Python, FastAPI}
  \item{Panel administracyjny - JavaScript, Svelte}
  \item{Broker MQTT - Python}
  \item{Terminal - Python}
\end{itemize}
\subsection{Najważniejsze funkcje systemu}
\subsubsection{Ustalanie trasy przez kierowcę}
\begin{lstlisting}[language={Python}, caption={Terminal, Lokalizacja: \texttt{client/client.py}}]
def begin_route() -> None:
  global stop_rfid_polling
  global current_stop_index
  global routes
  global route_index
  current_stop_index = 0
  GPIO.add_event_detect(buttonGreen, GPIO.FALLING, callback=on_green_button_while_on_route, bouncetime=500)
  GPIO.add_event_detect(buttonRed, GPIO.FALLING, callback=on_red_button_while_on_route, bouncetime=500)
  client.publish("buses/driver", f"choose_course?bus={bus_id}&course={routes[route_index].name}")
  draw_stops_screen(routes[route_index], current_stop_index)
  stop_rfid_polling = False
  rfid_thread = threading.Thread(target=listen_rfid)
  rfid_thread.daemon = True
  rfid_thread.start()
\end{lstlisting}
Po wybraniu trasy przez kierowcę terminal komunikuje się z brokerem MQTT poprzez temat \verb|buses/driver| i wysyła mu odpowiednie dane, następnie też aktywowana jest możliwość skanowania karty pracownika zaimplementowana w wątku \verb|rfid_thread|. Dodatkowo funkcjonalność przycisków: zielonego i czerwonego zostaje zmieniona, tak aby była odpowiedzialna już za zmianę przystanków, a wyświetlacz pokazuje przystanki.
\begin{lstlisting}[language={Python}, caption={Broker MQTT, Lokalizacja: \texttt{backend/mqtt\_server.py}}]
def choose_course(bus_id: int, course_name: str):
  course_id = db_management.select('Courses', ['CourseID'], [('CourseName', course_name)])[0][0]
  db_management.update('Buses', ('CourseID', course_id), ('BusID', bus_id))
  db_management.update('Buses', ('StopNumber', 1), ('BusID', bus_id))
\end{lstlisting}
Broker MQTT po otrzymaniu danych od terminala, aktualizuje bazę danych.
\subsubsection{Zmiana przystanku przez kierowcę}
\begin{lstlisting}[language={Python}, caption={Terminal, Lokalizacja: \texttt{client/client.py}}]
def on_green_button_while_on_route(_) -> None:
  global current_stop_index
  global routes
  global route_index
  global stop_rfid_polling

  route = routes[route_index]
  client.publish("buses/driver", f"next_stop?bus={bus_id}")
  if current_stop_index < len(route.stops) - 1:
      current_stop_index += 1
      draw_stops_screen(route, current_stop_index)
  else:
      stop_rfid_polling = True
      GPIO.remove_event_detect(buttonGreen)
      GPIO.remove_event_detect(buttonRed)
      route_index = 0
      current_stop_index = 0
      select_route()
\end{lstlisting}
Przy zmianie przystanku przez kierowcę terminal wysyła dane do brokera MQTT, następnie informacja o aktualnym przystanku na wyświetlaczu zostaje zmieniona. Jeżeli następnego przystanku nie ma, to wtedy trasa jest automatycznie zakończona.
\begin{lstlisting}[language={Python}, caption={Broker MQTT, Lokalizacja: \texttt{backend/mqtt\_server.py}}]
def next_stop(bus_id: int):
  bus_data = db_management.select('Buses', ['BusID', 'CourseID', 'StopNumber'],
                                  [('BusID', bus_id)])
  if not bus_data:
      return NO_SUCH_BUS
  bus_id, course_id, stop_number = bus_data[0]
  try:
      stops = [tup[0] for tup in db_management.select('Assignments', ['StopID'],
                                                      [('CourseID', course_id)])]
  except OperationalError:
      return BUS_NOT_ON_ROUTE
  new_stop_number = stop_number + 1
  if new_stop_number == 0 or new_stop_number > len(stops):
      for attribute_name in ['CourseID', 'StopNumber']:
          db_management.update('Buses', (attribute_name, 'null'), ('BusID', bus_id))
      return ROUTE_ENDED
  else:
      db_management.update('Buses', ('stopNumber', new_stop_number), ('BusID', bus_id))
      new_stop_id = db_management.select('Assignments', ['StopID'],
                                         [('CourseID', course_id), ('stopNumber', new_stop_number)])[0][0]
      new_stop_name = db_management.select('Stops', ['StopName'], [('StopID', new_stop_id)])[0][0]
      add_stop_to_workers(bus_id)
      return f'{new_stop_number}-{new_stop_name}'
\end{lstlisting}
Broker MQTT po otrzymaniu danych od terminala aktualizuje przystanek w bazie danych, jeżeli podany pojazd jest na trasie, następnie aktualizowana jest liczba przejechanych przystanków dla każdego pracownika który znajduje się na trasie pojazdu.
\subsubsection{Płacenie za przejazd przez pracownika}
\begin{lstlisting}[language={Python}, caption={Terminal, Lokalizacja: \texttt{client/client.py}}]
def on_card_scanned(uid: list[int]) -> None:
  global last_card_scan_value_time
  global current_stop_index
  global routes
  global route_index
  uid_int = int(''.join(list(map(lambda e: str(e), uid))))
  print(f'scanned {uid_int}')
  (last_value, last_time) = last_card_scan_value_time
  if last_time is not None and last_time + datetime.timedelta(seconds=7) > datetime.datetime.now():
      return
  last_card_scan_value_time = (uid_int, datetime.datetime.now())
  client.publish("buses/worker", f"use_card?card={uid_int}&bus={bus_id}")
\end{lstlisting}
Po przyłożeniu karty przez pracownika, terminal wysyła dane do brokera MQTT.
\pagebreak
\begin{lstlisting}[language={Python}, caption={Broker MQTT, Lokalizacja: \texttt{backend/mqtt\_server.py}}]
def card_used(card_id: str, bus_id: int):
  riding_workers = [tup[0] for tup in db_management.select_all('CurrentRides', ['WorkerCardID'])]
  if int(card_id) in riding_workers:
      return worker_gets_out(card_id)
  else:
      return worker_gets_in(card_id, bus_id)


def worker_gets_in(card_id: str, bus_id: int):
  if int(card_id) not in [tup[0] for tup in db_management.select_all('Workers', ['WorkerCardID'])]:
      return 'Invalid card.'
  try:
      max_ride_id = max([tup[0] for tup in db_management.select_all('CurrentRides', ['RideID'])])
  except ValueError:
      max_ride_id = 0
  db_management.insert('CurrentRides', (max_ride_id + 1, card_id, bus_id, 0))
  return f'Card validated succesfully.'


def worker_gets_out(card_id: str):
  stops_traveled = db_management.select('CurrentRides', ['StopsTraveled'], [('WorkerCardID', card_id)])[0][0]
  db_management.delete('CurrentRides', ('WorkerCardID', card_id))
  current_balance = db_management.select('Workers', ['WorkerBalance'], [('WorkerCardID', card_id)])[0][0]
  db_management.update('Workers', ('WorkerBalance', current_balance - stops_traveled), ('WorkerCardID', card_id))
  return f'Balance after ride: {current_balance - stops_traveled}.'
\end{lstlisting}
Broker MQTT po otrzymaniu danych od terminala, sprawdza czy pracownik zaczyna, lub kończy przejazd:
\begin{itemize}
  \item{jeżeli zaczyna przejazd, to po sprawdzeniu czy taki pracownik istnieje w bazie, zostaje dodany rekord o aktualnym przejeździe,}
  \item{jeżeli kończy przejazd, to z jego konta zostaje potrącona kwota zależna od przejechanych przystanków, oraz jego przejazd zostaje usunięty z bazy.}
\end{itemize}
\subsubsection{Dodanie pieniędzy do konta pracownika przez administratora}
\begin{lstlisting}[language={JavaScript}, caption={Panel administracyjny, Lokalizacja: \texttt{frontend/app/src/routes/employees/+page.svelte}}]
async function editBalance(){
    if(bonusData.bonus == null) bonusData.bonus = 0.0
    try{
        const url = `/addbalance/${bonusData.worker_id}?value=${bonusData.bonus}`
        const response = await fetch(baseUrl + url, {
            method: "GET",
        })
        if(response.ok){
            console.log("Form data sent successfully");
            employees = await getEmployeesData();
        }else{
            console.error("error sending form data", response.status);
        }
    }catch(e){
        console.log(e);
    }
}
\end{lstlisting}
Panel administracyjny przekazuje odpowiednie dane do REST API, następnie pobiera zaktualizowane dane o pracownikach.
\begin{lstlisting}[language={Python}, caption={REST API, Lokalizacja: \texttt{backend/server.py}}]
@app.get("/addbalance/{worker_id}")
async def add_balance_endpoint(worker_id: int, value: float):
    current_balance = db_management.select('Workers', ['WorkerBalance'], [('WorkerID', worker_id)])[0][0]
    db_management.update('Workers', ('WorkerBalance', current_balance + value), ('WorkerID', worker_id))
    return {'success': f'Balance of worker with ID {worker_id} has changed from {current_balance} to '
                        f'{current_balance + value}'}
\end{lstlisting}
REST API po otrzymaniu danych, aktualizuje bazę danych.
\subsection{Implementacja MQTT}
\subsubsection{Broker MQTT}
\begin{lstlisting}[language={Python}, caption={Broker MQTT, Lokalizacja: \texttt{backend/mqtt\_server.py}}]
import paho.mqtt.client as mqtt
(...)

broker_ip = "0.0.0.0"

CLIENT = mqtt.Client()

(...)


def process_message(client, userdata, message):
    global CLIENT
    message_decoded = (str(message.payload.decode("utf-8"))).split(".")[0]
    print(message_decoded)
    message_dict = query_string_to_dict(message_decoded)
    if 'next_stop' in message_dict:
        next_stop(message_dict['next_stop']['bus'])
    elif 'choose_course' in message_dict:
        choose_course(message_dict['choose_course']['bus'], message_dict['choose_course']['course'])
    elif 'use_card' in message_dict:
        result_code = card_used(message_dict['use_card']['card'], message_dict['use_card']['bus'])
        CLIENT.publish('response/success', result_code)


def connect_to_broker():
    CLIENT.connect(broker_ip)
    CLIENT.on_message = process_message
    CLIENT.loop_start()
    CLIENT.subscribe("buses/#")


def disconnect_from_broker():
    CLIENT.loop_stop()
    CLIENT.disconnect()


def run_mqtt_server():
    connect_to_broker()
    input()


(...)


if __name__ == "__main__":
    run_mqtt_server()
\end{lstlisting}
Broker MQTT subskrybuje każdy temat, który zaczyna się od \verb|buses/|, każda wiadomość która jest odbierana przez broker jest w formacie \verb|query string|, natomiast odpowiedzi są wysyłane tematem \verb|response/success|.
\pagebreak
\subsubsection{Terminal}
\begin{lstlisting}[language={Python}, caption={Terminal, Lokalizacja: \texttt{client/client.py}}]
#!/usr/bin/env python3

import paho.mqtt.client as mqtt
(...)

server_ip = "10.108.33.123"
bus_id = 0

client = mqtt.Client()
(...)


def begin_route() -> None:
    (...)
    client.publish("buses/driver", f"choose_course?bus={bus_id}&course={routes[route_index].name}")
    (...)


def on_mqtt_message(client, userdata, message):
    message_decoded = str(message.payload.decode('utf-8'))
    print(f'message received: {message_decoded}')
    draw_message(message_decoded)
    timer = threading.Timer(7, draw_stops_screen, args=(routes[route_index], current_stop_index))
    timer.start()


(...)


if __name__ == "__main__":
    try:
        disp.Init()
        disp.clear()

        client.connect(server_ip)
        client.on_message = on_mqtt_message
        client.loop_start()
        client.subscribe("response/#")

        (...)

        while True:
            _ = input()
    (...)
\end{lstlisting}
Terminal subskrybuje wszystkie tematy zaczynające się od \verb|response/|, oraz publikuje dane na różnych tematach zaczynających się od \verb|buses/|. Dane które są odebrane z \verb|response/| są przekazywane do wyświetlacza, który je pokazuje.
\subsection{Implementacja terminala}
\subsubsection{Pobieranie danych o kursach i ich przystankach}
\begin{lstlisting}[language={Python}, caption={Terminal, Lokalizacja: \texttt{client/client.py}}]
def fetch_routes() -> list[Route]:
  with urllib.request.urlopen(f'http://{server_ip}:55555/courses') as url:
      data = json.load(url)
      for route_name in data:
          stops = map(lambda stop_name: Stop(stop_name), data[route_name])
          routes.append(Route(route_name, list(stops)))
  return routes
\end{lstlisting}
Do pobierania danych o kursach i przystankach terminal wykorzystuje REST API.
\pagebreak
\begin{lstlisting}[language={Python}, caption={REST API, Lokalizacja: \texttt{backend/server.py}}]
@app.get("/courses")
async def courses_endpoint():
    result = {}
    courses = db_management.select_all('Courses', ['CourseID', 'courseName'])
    for course_id, course_name in courses:
        stops = db_management.select('Assignments', ['StopID', 'StopNumber'], [('CourseID', course_id)])
        result[course_name] = ['' for _ in range(len(stops))]
        for stop_id, stop_number in stops:
            stop_name = db_management.select('Stops', ['StopName'], [('StopID', stop_id)])[0][0]
            result[course_name][stop_number-1] = stop_name
    return result
\end{lstlisting}
REST API pobiera z bazy danych kursy, następnie pobiera ich przystanki i je przypisuje do zwracanych danych.
