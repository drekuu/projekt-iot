\section{Opis architektury systemu}
\subsection{Elementy architektury z opisem}
\begin{itemize}
  \item{Panel administracyjny - służy administratorom do zarządzania danymi}
  \item{Baza danych - zawiera wszystkie dane}
  \item{REST API - zarządza danymi bazy danych i udostępnia odpowiednie operacje innym podmiotom}
  \item{Broker MQTT - komunikuje się z terminalami w pojazdach i przekazuje informacje o przejechanych przystankach i płatnościach}
  \item{Terminal - znajduje się w każdym pojeździe, umożliwia kierowcy ustalanie trasy i zmianę przystanków, oraz umożliwia pracownikom płacenie za swoje przejazdy}
  \item{
    Użytkownicy
    \begin{itemize}
      \item{Kierowca - kieruje pojazdem i na terminalu może zmieniać trasy, przystanki}
      \item{Pracownik - może wsiadać do pojazdów i płacić w terminalu za przejazd kartą}
      \item{Administrator - zarządza danymi w systemie}
    \end{itemize}
  }
\end{itemize}
\subsection{Graficzna reprezentacja architektury}
\begin{figure}[H]
  \centering
  \begin{tikzpicture}[>=Latex,
    user/.style={draw=gray,dashed}
  ]
    \node at (0,0) [rectangle,draw] (database) {Baza danych};
    \node [below=of database,rectangle,draw] (rest) {REST API};
    \node [below left=of rest,rectangle,draw] (frontend_admin) {Panel administracyjny};
    \node [below right=of rest,rectangle,draw] (broker) {Broker MQTT};
    \node [below=of broker,rectangle,draw] (terminal) {Terminal};
    \node [below left=of broker,rectangle] (terminaldotsl) {\dots};
    \node [below right=of broker,rectangle] (terminaldotsr) {\dots};

    \node[below=0.5 of frontend_admin,rectangle,draw,user] (admin) {Administrator};
    \node[below left=0.5 of terminal,rectangle,draw,user] (driver) {Kierowca};
    \node[below right=0.5 of terminal,rectangle,draw,user] (worker) {Pracownik};

    \draw [<->] (database) -- (rest);
    \draw [<->] (rest) -- (frontend_admin);
    \draw [<->] (rest) -- (broker);
    \draw [<->] (broker) -- (terminal);
    \draw [<->] (broker) -- (terminaldotsl);
    \draw [<->] (broker) -- (terminaldotsr);

    \draw [<->] (admin) -- (frontend_admin);
    \draw [<->] (driver) -- (terminal);
    \draw [<->] (worker) -- (terminal);
  \end{tikzpicture}
  \caption{Diagram elementów architektury z kierunkami przekazywania danych}
\end{figure}
\subsection{Baza danych}
\subsubsection{Schemat bazy danych}
\begin{figure}[H]
  \centering
  \begin{tikzpicture}[>=Latex,
    db/.style={
      draw, matrix of nodes,
      nodes={
        node family/text width/.expanded=%
          \tikzmatrixname-\the\pgfmatrixcurrentcolumn,
        node family/text width align=left,
        inner xsep=+.5\tabcolsep, inner ysep=+0pt, align=left},
      inner sep=.5\pgflinewidth,
      font=\strut\ttfamily,
    }
  ]
    \matrix[db,label=Courses,row 1/.style={nodes={fill=red!20}}] (courses) {
      CourseID & int & \node (courses_courseid) {PK}; \\
      CourseName & varchar(255) \\
    };
    \matrix[db,label=Stops,below=of courses,row 1/.style={nodes={fill=red!20}}] (stops) {
      StopID & int & \node (stops_stopid) {PK}; \\
      StopName & varchar(255) \\
    };
    \matrix[db,label=Assignments,below=of stops,row 1/.style={nodes={fill=blue!20}},row 2/.style={nodes={fill=blue!20}}] (assignments) {
      CourseID & int & \node (assignments_courseid) {FK}; \\
      StopID & int & \node (assignments_stopid) {FK}; \\
      StopNumber & int \\
    };
    \matrix[db,label=Workers,below=of assignments,row 1/.style={nodes={fill=red!20}}] (workers) {
      WorkerID & int & \node (workers_workerid) {PK}; \\
      WorkerFirstName & varchar(255) \\
      WorkerLastName & varchar(255) \\
      WorkerBalance & int \\
      WorkerCardID & varchar(255) \\
    };
    \matrix[db,label=CurrentRides,below=of workers,row 1/.style={nodes={fill=red!20}},row 2/.style={nodes={fill=blue!20}}] {
      RideID & int & PK \\
      WorkerID & int & \node (currentrides_workerid) {FK}; \\
      StopsTraveled & int \\
    };
    \draw [<-] (courses_courseid.east) -- ++(0.5,0) |- (assignments_courseid.east);
    \draw [<-] (stops_stopid.east) -- ++(0.5,0) |- (assignments_stopid.east);
    \draw [<-] (workers_workerid.east) -- ++(0.5,0) |- (currentrides_workerid.east);
  \end{tikzpicture}
  \caption{Tabele bazy danych z zaznaczonymi relacjami między nimi}
\end{figure}
\subsubsection{Scenariusze i ich wpływ na dane}
\begin{enumerate}
  \item{
    Pracownik wsiada na przystanku, przykłada kartę i po przejechaniu 3 przystanków przykłada ją znowu aby zapłacić i wysiada. \\
    Zakładamy, że każdy przystanek kosztuje 10, czyli pracownik zapłaci 30.
    \begin{figure}[H]
      \centering
      \begin{tikzpicture}[>=Latex,
        db/.style={
          draw, matrix of nodes,
          nodes={
            node family/text width/.expanded=%
              \tikzmatrixname-\the\pgfmatrixcurrentcolumn,
            node family/text width align=left,
            inner xsep=+.5\tabcolsep, inner ysep=+0pt, align=left},
          inner sep=.5\pgflinewidth,
          font=\strut\ttfamily,
          row 1/.style={nodes={fill=gray!20}}
        }
      ]
        \matrix[db,label=Workers] (workers) {
          WorkerID & WorkerFirstName & WorkerLastName & WorkerBalance & WorkerCardID \\
          1 & Jan & Kowalski & 100 & ABC \\
        };
      \end{tikzpicture}
      \caption{Dane w bazie przed wykonaniem scenariusza}
    \end{figure}
    \begin{figure}[H]
      \centering
      \begin{tikzpicture}[>=Latex,
        db/.style={
          draw, matrix of nodes,
          nodes={
            node family/text width/.expanded=%
              \tikzmatrixname-\the\pgfmatrixcurrentcolumn,
            node family/text width align=left,
            inner xsep=+.5\tabcolsep, inner ysep=+0pt, align=left},
          inner sep=.5\pgflinewidth,
          font=\strut\ttfamily,
          row 1/.style={nodes={fill=gray!20}}
        }
      ]
        \matrix[db,label=Workers] (workers) {
          WorkerID & WorkerFirstName & WorkerLastName & WorkerBalance & WorkerCardID \\
          1 & Jan & Kowalski & 70 & ABC \\
        };
        \matrix[db,label=CurrentRides,below=of workers] {
          RideID & WorkerID & StopsTraveled \\
          \vdots \\
          5 & 1 & 3 \\
          \vdots \\
        };
      \end{tikzpicture}
      \caption{Dane w bazie po wykonaniu scenariusza}
    \end{figure}
    W tabeli \verb|Workers| zmieniła się kolumna \verb|WorkerBalance|, a w tabeli \verb|CurrentRides| został dodany nowy rekord.
  }
\end{enumerate}
